\chapter{PID - Regler}

\section{P - regulator}

Mathematisch kann das P - regulator beschreiben als:
\begin{equation}
	u = K e(t)
\end{equation}
Durch ein Pseudo-code kann das Regler beschreiben als:
\begin{equation}
	output := gain * error;
\end{equation}
Weil ein P-Regler ist immer durch der Fehler beschriebt deshalb, f\"uhrt es zu steady-state Fehler. Dies steady-state Fehler kann durch sequentielle Addition beseitigt (eliminate) werden. Die sequentielle Addition kann \"uber eine integrierte Regler erfolgen.
