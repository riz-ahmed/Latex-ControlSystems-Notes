%
%
% Headerdatei der Diplomarbeit
%
%
\documentclass[12pt, pdftex, pointlessnumbers, twoside, openright]{scrreprt}


\usepackage{a4}									% a4.sty Datei wird z. B. mit dem package ntgclass installiert. 
% Man kann die a4.sty auch so herunterladen und in den Projektordner stellen
\usepackage{ae}
\usepackage{amsbsy}
\usepackage{amsfonts}
\usepackage{amsmath}
\usepackage{amssymb}						% amssymb,amsmath: Viele zus?tzliche Mathesymbole, siehe Anleitung
\usepackage{amsthm}
\usepackage{array}
\usepackage[english]{babel}
\usepackage{blindtext}
\usepackage{bibgerm}
\usepackage{color}
\usepackage{epsfig}
\usepackage{float}
\usepackage[T1]{fontenc}
\usepackage{graphicx}
\usepackage[latin1]{inputenc}
\usepackage{latexsym}
\usepackage{layout}
\usepackage[german]{nomencl}		% nomencl package f?r Symbol- oder Formelverzeichnis
\usepackage{scrlayer-scrpage}
\usepackage{subfigure}					% subfigure: Mehrere Unterbilder in einem Bild
\usepackage{times}
\usepackage{trfsigns}
\usepackage{trsym}
\usepackage[nice]{units}				% units: Setzt Zahlen mit Einheiten im Math- und im Textmodus nach Din 1338 verwenden mit 
% \unit[Zahl]{Einheit} oder \unitfrac[Zahl]{Einheitz?hler}{Einheitnenner} oder 
% \nicefrac[schrift]{z?hler}{nenner}
\usepackage{url}

\usepackage{listings}

% \usepackage{epstopdf}
%	\usepackage[dvips]{geometry}
% \usepackage{pdflscape} 					% pdflscape: Einzelne Seiten im Querformat unter PDF
% \usepackage[dvips]{rotating}		% rotating: Drehen von Text und Bildern


\newcommand{\MB}[1]{{\mbox{\mathversion{bold}$#1$}}}					% Befehl schreibt griechische Buchstaben fett
\newcommand*\parfr[2]{\frac{\partial {#1}}{\partial {#2}}}		% partielle Ableitung mit Bruch
%\newcommand{\vec}[1]{\boldsymbol{#1}} 											  % Vektoren/Matizen fett - Variable
\newcommand{\mat}[1]{\begin{pmatrix}#1\end{pmatrix}} 					% Matrix


\renewcommand{\nomname}{Nomenclature}


\allowdisplaybreaks


\makenomenclature 							% makenomeclature: Verwenden zum Erstellen eines Symbolverzeichnisses, 
% Kann z. B. mit Formelzeichen verwendet werden


%-----------------------HYPERREF

\usepackage[pdftex, colorlinks=false, plainpages=false, pdfpagelabels]{hyperref}
\hypersetup{
	pdftitle = {Unbekannter Titel},
	pdfsubject = {Masterarbeit am Lehrstuhl f\"ur Regelungssysteme, Fachbereich EIT, TU Kaiserslautern},
	pdfauthor = {Rizwan Ahmed Afzal},
	pdfkeywords = {},
	pdfcreator = {pdftex},
	pdfproducer = {Rizwan Ahmed}
}

%------------------------END HYPERREF


% Mit \Version kann das Datum eingef?gt werden
% \newcommand{\VERSION}{\today}


% Nach DIN 1338 m?ssen Gleichungen in Klammern zitiert werden, z.B: siehe Gleichung (2.15)
% mit \eqref{ref} werden die Klammen automatisch gesetzt


\setkomafont{sectioning}{\rmfamily\bfseries}		% Umschaltung auf ?berschriften mit Serifen


% \theoremstyle{plain}
% \theoremstyle{remark}
% \theoremstyle{definition}
\newtheorem{Def}{Definition}
\newtheorem{lem}{Lemma}
\newtheorem{satz}{Satz}
\newtheorem{prob}{Problem}

\newcommand{\norm}[1]{\lVert#1\rVert}
\renewcommand{\vec}[1]{\mathbf{#1}}

%Image-related packages
\usepackage{graphicx,subfigure}
%\usepackage{subcaption}
%\usepackage[utf8]{inputenc}
\usepackage[export]{adjustbox}
\usepackage{wrapfig,lipsum}

\usepackage{graphicx} 
\usepackage{color} 
\usepackage{transparent}

\usepackage{array}
\usepackage{booktabs}

\usepackage{mathtools}

% \usepackage[toc,page]{appendix}             % For writing appendices


\usepackage{xcolor}
\usepackage{soul}
\newcommand{\mathcolorbox}[2]{\colorbox{#1}{$\displaystyle #2$}}